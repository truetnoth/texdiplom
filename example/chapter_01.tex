\Section{Раздел 1}

\mainText

\Subsection{подраздел 1}
Предприятием-заказчиком является Факультет среднего профессионального образования Университета ИТМО. Факультет среднего профессионального образования был основан в 1945 году. За свою историю ФСПО несколько раз меняло свой профиль, отслеживая тенденции развития промышленности, сначала с направления механики на приборостроение, затем на информационные технологии. Актуальность последнего направления привела к выводу о целесообразности сближения с университетским образованием и в 1998 году техникум вошел в состав университета, а в 2003 году по решению Ученого Совета был преобразован в факультет. Его организационная структура организации представлена на рисунке 1.
\Subsection{SWOT-анализ}
SWOT-анализ – это метод стратегического планирования, заключающийся в выявлении факторов внутренней и внешней среды организации и разделении их на четыре категории:
\begin{enumerate}
	\gostEnum
	\item «Strengths» (сильные стороны);
	\item «Weaknesses» (слабые стороны);
	\item «Opportunities» (возможности);
	\item «Threats» (угрозы).
\end{enumerate}

Сформулированы факторы, оказывающие воздействие:
\par
\begin{enumerate}
	\gostEnumPriority
	\item	Внутренние факторы:
	\begin{enumerate}
		\gostEnumPriority
		\item качество преподавания;
		\item эффективность проведения практических занятий;
		\item техническое оснащение;
		\item внеучебная деятельность;
		\item стоимость обучения.		
	\end{enumerate}
	\item Внешние факторы:
	\begin{enumerate}
		\gostEnumPriority
		\item Динамика количества абитуриентов
		\item Участие в мероприятиях за пределами факультета
		\item Технический прогресс
	\end{enumerate}

\end{enumerate}

Как известно,

\begin{equation}
\label{trivial}
7\times9=63.
\end{equation}
\begin{equation}
\label{trivial}
5\times9=45.
\end{equation}

Из проведенного выше SWOT – анализа можно выявить следующую проблему: организация заказчика нуждается в таком программном продукте (ПП), который смог бы уменьшить время решения рутинных и однотипных задач во время проведения практических занятий, тем самым повышая заинтересованность студентов. Разрабатываемый ПП сможет решить эту проблему, что впоследствии положительно повлияет следующие стороны организации:
\begin{enumerate}
	\gostEnum
	\item эффективность проведения практических занятий;
	\item качество преподавания.
\end{enumerate}


	\begin{longtable}{|l|l|l|}
		\gostTable{Feasible triples for highly variable Grid, MLMMH.}
		\label{grid_mlmmh} 
		\\ 	\hline 
		Time (s) & Triple chosen \\
		\hline \endfirsthead
			\gostLongTable{\thetable{}}
		\\ \hline \endhead
		\endfoot
		\hline
		\endlastfoot
		0 & (1, 11, 13725)\\
		2745 & (1, 12, 10980) \\
		5490 & (1, 12, 13725) \\
		8235 & (1, 12, 16470) \\
		10980 & (1, 12, 16470) \\
		13725 & (1, 12, 16470) \\
		16470 & (1, 13, 16470) \\
		19215 & (1, 12, 16470) \\
		21960 & (1, 12, 16470) \\
		24705 & (1, 12, 16470) \\
		27450 & (1, 12, 16470) \\
		30195 & (2, 2, 2745) \\
		32940 & (1, 13, 16470) \\
		35685 & (1, 13, 13725) \\
		38430 & (1, 13, 10980) \\
		41175 & (1, 12, 13725) \\
		43920 & (1, 13, 10980) \\
		46665 & (2, 2, 2745) \\
		49410 & (2, 2, 2745) \\
		52155 & (1, 12, 16470) \\
		54900 & (1, 13, 13725) \\
		57645 & (1, 13, 13725) \\
		60390 & (1, 12, 13725) \\
		63135 & (1, 13, 16470) \\
		65880 & (1, 13, 16470)\\
		68625 & (2, 2, 2745) \\
		71370 & (1, 13, 13725) \\
		74115 & (1, 12, 13725) \\
		76860 & (1, 13, 13725) \\
		79605 & (1, 13, 13725) \\
		82350 & (1, 12, 13725) \\
		85095 & (1, 12, 13725) \\
		87840 & (1, 13, 16470) \\
		90585 & (1, 13, 16470) \\
		93330 & (1, 13, 13725) \\
		96075 & (1, 13, 16470) \\
		98820 & (1, 13, 16470) \\
		101565 & (1, 13, 13725) \\
		104310 & (1, 13, 16470) \\
		107055 & (1, 13, 13725) \\
		109800 & (1, 13, 13725) \\
		112545 & (1, 12, 16470)  \\
		115290 & (1, 13, 16470) \\
		118035 & (1, 13, 13725) \\
		120780 & (1, 13, 16470) \\
		123525 & (1, 13, 13725) \\
		126270 & (1, 12, 16470) \\
		129015 & (2, 2, 2745) \\
		131760 & (2, 2, 2745)  \\
		134505 & (1, 13, 16470) \\
		137250 & (1, 13, 13725)\\
		139995 & (2, 2, 2745)  \\
		142740 & (2, 2, 2745)  \\
		145485 & (1, 12, 16470) \\
		148230 & (2, 2, 2745)  \\
		150975 & (1, 13, 16470)  \\
		153720 & (1, 12, 13725) \\
		156465 & (1, 13, 13725)  \\
		159210 & (1, 13, 13725)  \\
		161955 & (1, 13, 16470)  \\
		164700 & (1, 13, 13725) \\
		109800 & (1, 13, 13725) \\
		112545 & (1, 12, 16470)  \\
		115290 & (1, 13, 16470) \\
		118035 & (1, 13, 13725) \\
		120780 & (1, 13, 16470) \\
		123525 & (1, 13, 13725) \\
		126270 & (1, 12, 16470) \\
		129015 & (2, 2, 2745) \\
		131760 & (2, 2, 2745)  \\
		134505 & (1, 13, 16470) \\
		137250 & (1, 13, 13725)\\
		139995 & (2, 2, 2745)  \\
		142740 & (2, 2, 2745)  \\
		145485 & (1, 12, 16470) \\
		148230 & (2, 2, 2745)  \\
		150975 & (1, 13, 16470)  \\
		153720 & (1, 12, 13725) \\
		156465 & (1, 13, 13725)  \\
		159210 & (1, 13, 13725)  \\
		161955 & (1, 13, 16470)  \\
		164700 & (1, 13, 13725) \\
		109800 & (1, 13, 13725) \\
		112545 & (1, 12, 16470)  \\
		115290 & (1, 13, 16470) \\
		118035 & (1, 13, 13725) \\
		120780 & (1, 13, 16470) \\
		123525 & (1, 13, 13725) \\
		126270 & (1, 12, 16470) \\
		129015 & (2, 2, 2745) \\
		131760 & (2, 2, 2745)  \\
		134505 & (1, 13, 16470) \\
		137250 & (1, 13, 13725)\\
		139995 & (2, 2, 2745)  \\
		142740 & (2, 2, 2745)  \\
		145485 & (1, 12, 16470) \\
		148230 & (2, 2, 2745)  \\
		150975 & (1, 13, 16470)  \\
		153720 & (1, 12, 13725) \\
		156465 & (1, 13, 13725)  \\
		159210 & (1, 13, 13725)  \\
		161955 & (1, 13, 16470)  \\
		164700 & (1, 13, 13725) \\
	\end{longtable}

\begin{enumerate}
	\gostEnum
	\item это первое в списке:
		\begin{enumerate}
			\gostEnum
			\item где порядок;
			\item первого уровня;
			\item не важен
		\end{enumerate}
	\item а это второе;
	\item и последнее.
\end{enumerate}

Curabitur feugiat porta lectus, ac fermentum urna iaculis at. Ut a iaculis enim, quis fringilla dui. Cras scelerisque nisi a maximus maximus. Suspendisse luctus luctus porttitor. In mauris mi, vehicula sed hendrerit in, ultrices sed velit. Phasellus posuere ultricies porttitor. Maecenas sit amet felis lacinia, eleifend mi id, scelerisque orci. Morbi lorem ipsum, mollis at metus a, ullamcorper ullamcorper felis.


\begin{enumerate}	
	\gostEnumPriority
	\item это первое в списке
	\begin{enumerate}
		\gostEnumPriority
		\item где важен;
		\item порядок;
		\item первого уровня;
	\end{enumerate}
	
	\item а это второе;
	\item и последнее;
\end{enumerate}
\begin{longtable}{|l|l|l|}
		\gostTable{Пример длинной таблицы с длинным названием на много длинных-длинных строк}
		\label{tab:longtable}
		\\ \hline
		Вид шума & Громкость, дБ  \\
		\hline \endfirsthead
		\gostLongTable{\thetable{}}
		\\ \hline \endhead
		\endfoot
		\hline \endlastfoot
		Порог слышимости             & 0   \\
		\hline
		Шепот в тихой библиотеке     & 30  \\
		Обычный разговор             & 60-70\\
		Звонок телефона              & 80   \\
		Уличный шум                  & 85  \\
		Гудок поезда                 & 90   \\
		Шум электрички               & 95   \\
		\hline
		Порог здоровой нормы         & 90-95 \\
		\hline
		Мотоцикл                     & 100   \\
		Power Mower                  & 107   \\
		Бензопила                    & 110   \\
		Рок-концерт                  & 115   \\
		\hline
		Порог боли                   & 125   \\
		\hline
		Клепальный молоток           & 125   \\
		\hline
		Порог опасности              & 140                          \\
		\hline
		Реактивный двигатель         & 140   \\
		& 180                                     \\
		Самый громкий возможный звук & 194                   \\
	\end{longtable}

\begin{longtable}{|l|l|l|}
		\gostTable{Feasible triples for highly variable Grid, MLMMH.} \label{grid_mlmmh} \\
		
		\hline \multicolumn{1}{|c|}{\textbf{Time (s)}} & \multicolumn{1}{c|}{\textbf{Triple chosen}} \\ \hline 
		\endfirsthead

		\gostLongTable{\thetable{}}\\
		%\gostLongTable \\
		%{\thetable{}}\\
		\hline \multicolumn{1}{|c|}{\textbf{Time (s)}} &
		\multicolumn{1}{c|}{\textbf{Triple chosen}} \\ \hline 
		\endhead
		
		\endfoot
		
		\hline
		\endlastfoot
		
		0 & (1, 11, 13725)\\
		2745 & (1, 12, 10980) \\
		5490 & (1, 12, 13725) \\
		8235 & (1, 12, 16470) \\
		10980 & (1, 12, 16470) \\
		13725 & (1, 12, 16470) \\
		16470 & (1, 13, 16470) \\
		19215 & (1, 12, 16470) \\
		21960 & (1, 12, 16470) \\
		24705 & (1, 12, 16470) \\
		27450 & (1, 12, 16470) \\
		30195 & (2, 2, 2745) \\
		32940 & (1, 13, 16470) \\
		35685 & (1, 13, 13725) \\
		38430 & (1, 13, 10980) \\
		41175 & (1, 12, 13725) \\
		43920 & (1, 13, 10980) \\
		46665 & (2, 2, 2745) \\
		49410 & (2, 2, 2745) \\
		52155 & (1, 12, 16470) \\
		54900 & (1, 13, 13725) \\
		57645 & (1, 13, 13725) \\
		60390 & (1, 12, 13725) \\
		63135 & (1, 13, 16470) \\
		65880 & (1, 13, 16470)\\
		68625 & (2, 2, 2745) \\
		71370 & (1, 13, 13725) \\
		74115 & (1, 12, 13725) \\
		76860 & (1, 13, 13725) \\
		79605 & (1, 13, 13725) \\
		82350 & (1, 12, 13725) \\
		85095 & (1, 12, 13725) \\
		87840 & (1, 13, 16470) \\
		90585 & (1, 13, 16470) \\
		93330 & (1, 13, 13725) \\
		96075 & (1, 13, 16470) \\
		98820 & (1, 13, 16470) \\
		101565 & (1, 13, 13725) \\
		104310 & (1, 13, 16470) \\
		107055 & (1, 13, 13725) \\
		109800 & (1, 13, 13725) \\
		112545 & (1, 12, 16470)  \\
		115290 & (1, 13, 16470) \\
		118035 & (1, 13, 13725) \\
		120780 & (1, 13, 16470) \\
		123525 & (1, 13, 13725) \\
		126270 & (1, 12, 16470) \\
		129015 & (2, 2, 2745) \\
		131760 & (2, 2, 2745)  \\
		134505 & (1, 13, 16470) \\
		137250 & (1, 13, 13725)\\
		139995 & (2, 2, 2745)  \\
		142740 & (2, 2, 2745)  \\
		145485 & (1, 12, 16470) \\
		148230 & (2, 2, 2745)  \\
		150975 & (1, 13, 16470)  \\
		153720 & (1, 12, 13725) \\
		156465 & (1, 13, 13725)  \\
		159210 & (1, 13, 13725)  \\
		161955 & (1, 13, 16470)  \\
		164700 & (1, 13, 13725) \\
		109800 & (1, 13, 13725) \\
		112545 & (1, 12, 16470)  \\
		115290 & (1, 13, 16470) \\
		118035 & (1, 13, 13725) \\
		120780 & (1, 13, 16470) \\
		123525 & (1, 13, 13725) \\
		126270 & (1, 12, 16470) \\
		129015 & (2, 2, 2745) \\
		131760 & (2, 2, 2745)  \\
		134505 & (1, 13, 16470) \\
		137250 & (1, 13, 13725)\\
		139995 & (2, 2, 2745)  \\
		142740 & (2, 2, 2745)  \\
		145485 & (1, 12, 16470) \\
		148230 & (2, 2, 2745)  \\
		150975 & (1, 13, 16470)  \\
		153720 & (1, 12, 13725) \\
	\end{longtable}
	

