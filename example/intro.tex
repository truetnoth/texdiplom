\Chapter{ВВЕДЕНИЕ}

\mainText
Актуальность темы курсового проектирования заключается в том, чтобы показать целесообразность разработки программного продукта (ПП) с экономической точки зрения и с точки зрения себестоимости продукта по сравнению с аналогичными решениями, существующими на рынке.
Целью курсового проектирования является экономическая оценка разработки.
Для достижения поставленной цели были выделены следующие задачи:
\gostEnumPriority
\begin{enumerate}
	\gostEnumPriority
	\item 	Дать организационно-правовую характеристику предприятию-заказчику
	\begin{enumerate}
		\gostEnumPriority
		\item Предоставить информацию о предприятии-заказчике
		\item Провести SWOT – анализ предприятия-заказчика
		\item Сделать обзор существующих аналогов программного продукта
	\end{enumerate}
	\gostEnumPriority 
	\item Описать планирование разработки программного продукта
	\begin{enumerate}
		\gostEnumPriority
		\item Составить календарный план-график
		\item Сформировать диаграмму Ганта с обоснованием возникших разрывов
	\end{enumerate}
	\item Оценить ресурсы, необходимые для разработки
	\begin{enumerate}
		\gostEnumPriority
		\item Оценить стоимости оборудования
		\item Оценить стоимости оборотных средств
		\item Рассчитать фонды оплаты труда
		\item Рассчитать взносы во внебюджетные фонды
		\item Рассчитать непредвиденные расходы
		\item Оценить себестоимость разработки программного продукта
	\end{enumerate}
\end{enumerate}

