\nChapter{РЕФЕРАТ}
\mainText
РАСХОДОМЕРНЫЕ УСТАНОВКИ, ПОРШНЕВЫЕ РАСХОДОМЕРЫ, ТАХОМЕТРИЧЕСКИЕ РАСХОДОМЕРЫ, ИЗМЕРЕНИЕ, БОЛЬШИЕ РАСХОДЫ, ГАЗЫ

Объектом исследования являются поршневые установки для точного воспроизведения и измерения больших расходов газа.

Цель работы - разработка методики метрологических исследований установок и нестандартной аппаратуры для их осуществления.

В процессе работы проводились экспериментальные исследования отдельных составляющих и общей погрешности установок.

В результате исследования впервые были созданы две поршневые реверсивные расходомерные установки: первая на расходы до 0,07 м /с, вторая - до 0,33 м /с.

Основные конструктивные и технико-эксплуатационные показатели: высокая точность измерения при больших значениях расхода газа.

Степень внедрения - вторая установка по разработанной методике аттестована как образцовая.

Эффективность установок определяется их малым влиянием на ход измеряемых процессов.

Обе установки могут применяться для градуировки и поверки промышленных ротационных счетчиков газа, а также тахометрических расходомеров.
